\documentclass[10pt, a4paper, twocolumn]{article}
\usepackage[T1]{fontenc} % Output font encoding for international characters
\usepackage[utf8]{inputenc} % Required for inputting international characters
\usepackage[ngerman]{babel}
\usepackage{XCharter} 
\usepackage{fancyhdr} 
\usepackage{geometry}
\usepackage{graphicx}
\usepackage{cite}
\usepackage{lipsum}
\usepackage[svgnames]{xcolor} 
\usepackage{booktabs} 
\usepackage{lastpage} 
\usepackage{enumitem} 

\usepackage{diffcoeff}
\usepackage{amsmath,amssymb,amsfonts}
\usepackage{algorithmic}
\usepackage{textcomp}
\usepackage{microtype}
\usepackage[hang, small, labelfont=bf, up, textfont=it]{caption} 
\usepackage{siunitx}
\usepackage{parskip}
\usepackage{sectsty} 

\newcommand{\abk}[1]{\left\langle #1 \right \rangle}
\newcommand{\bk}[1]{\left(#1\right)} 
\newcommand{\cbk}[1]{\left\{#1\right\}} 
\newcommand{\sbk}[1]{\left[#1\right]} 
\newcommand{\abs}[1]{\left|#1\right|} 
\newcommand{\norm}[1]{\left\|#1\right\|} 
\newcommand{\evec}[1]{\hat{\mathbf{#1}}}
%\renewcommand{\sin}[1]{\text{sin}\braket{#1}}
%\renewcommand{\cos}[1]{\cos\braket{#1}}
%\renewcommand{\tan}[1]{\tan\braket{#1}}
\newcommand{\R}{\mathbb{R}}
\newcommand{\N}{\mathbb{N}}
\newcommand{\Q}{\mathbb{Q}}
\renewcommand{\C}{\mathbb{C}}
\newcommand{\Z}{\mathbb{Z}}
\newcommand{\unit}[1]{\tag*{[\si{#1}]}}
\newcommand{\quer}[1]{\overline{#1}}

\DeclareMathOperator{\divergence}{div}
\DeclareMathOperator{\rot}{rot}
\DeclareMathOperator{\grad}{grad}

\setlist{noitemsep} 
\allsectionsfont{\usefont{OT1}{phv}{b}{n}}
\geometry{
	top=1cm,
	bottom=1.5cm, 
	left=2cm, 
	right=2cm, 
	includehead,
	includefoot,
}
\setlength{\columnsep}{7mm} % Column separation width  
\pagestyle{fancy} 
\renewcommand{\headrulewidth}{0.4pt} 
\renewcommand{\footrulewidth}{0.4pt} 
\renewcommand{\sectionmark}[1]{\markboth{#1}{}} 
% Headers
\rhead{\footnotesize\leftmark}
\lhead{\footnotesize\textit{Wellenlehre, Elektrizität \& Magnetismus}}
\rfoot{\footnotesize Page \thepage\ von \pageref{LastPage}}
\cfoot{}
\fancypagestyle{firstpage}{
	\fancyhf{}
    \renewcommand{\headrulewidth}{0.0pt} 
	\rfoot{\footnotesize Seite \thepage\ von \pageref{LastPage}} 
}
\usepackage{titling} 
\pretitle{
	\vspace{-40pt}
	\rule{\linewidth}{0.4pt}\par\vskip 10pt
	\huge\selectfont
}

\posttitle{\par\vskip 15pt \large \text{im Rahmen der Physik II Vorlesung FS 2021} \\ \text{von Prof. Rainer \textsc{Wallny}}} 
\predate{\vspace{-5pt} \large}
\postdate{\vspace{4pt} \\ 
\rule{\linewidth}{0.4pt}}
\setlength{\parindent}{0em}

\title{Wellenlehre, Elektrizität \& Magnetismus \\ Zusammenfassung} 
\date{\today}

\begin{document}
    \maketitle 
    \textit{Die Kapitel und Unterkapitel entsprechen denen vom Skript. Die mit * markierten Kapitel sind nicht Prüfungsrelevant.}
    \thispagestyle{firstpage} 
    
%\setlength{\abovedisplayskip}{1.5em}
%\setlength{\belowdisplayskip}{1.5em}
%\setlength{\abovedisplayshortskip}{1.0em}
%\setlength{\belowdisplayshortskip}{1.0em}
    
    \section{Wellen}

\subsection{Einleitung}
\begin{itemize}
    \item transversale Wellen (z.B. Seilwellen, schwingende Saite)
    \item longitudinale Wellen (z.B. Schallwellen)
\end{itemize}

\subsection{Wellentypen, -ausbreitung}
\subsubsection{Wellenfunktion \& Wellengleichung}\[\xi = \xi(x,t) = f(x \pm vt)\]
mit $x$ Raumkoordinate, $t$ Zeit, $f$ Wellenformfunktion, $v$ Phasengeschwindigkeit

Für alle Prozesse mit Wellencharakter gilt:
$$\diffp[2]\xi t = v^2 \diffp[2]\xi x$$
Eine allgemeine Lösung für die Wellengleichung ist der Form:
$$\xi(x,t) = f(x-vt) + g(x+vt)$$
für beliebige Funktionen $f,g$.

\subsubsection{Harmonische Welle}
\[\xi(x,t) = \xi_0 \sin(kx \pm \omega t) = \xi_0 e^{i(kx\pm \omega t)}\]
mit $k$ Wellenzahl/Wellenvektor, $\xi_0$ Amplitude, $\omega$ Kreisfrequenz

Es gelten die folgenden Beziehungen:
\begin{itemize}
    \item $v = f \cdot \lambda = \frac{\omega}{k}$ \hfill Phasengeschw. $[\si{\meter\per\second}]$
    \item $k = \frac{2\pi}{\lambda}$ \hfill Wellenzahl $[\si{\per\meter}]$
    \item $f = \nu = \frac{\omega}{2\pi}$ \hfill Frequenz $[\si{\hertz}]$
    \item $T = f^{-1} = \frac{2\pi}{\omega}$ \hfill Periode $[\si{\second}]$
\end{itemize}

\subsubsection{Transversale Wellen}
Für das Modell der Seilwelle erhalten wir für die Ausbreitungsgeschwindigkeit
$$v = \sqrt{\frac{S}{\rho}}$$
wobei $S$ mit $[\si{\newton\per\square\meter}]$ die Zugspannung und $\rho$ mit $[\si{\kilogram\per\meter\cubed}]$ die Dichte des Seiles ist.

\subsubsection{Longitudinale Wellen}
In einem Festkörper gilt für die \textbf{Normalspannung} $\sigma$ entlang des Mediums
\[\sigma = \diff {F_\perp} a \tag*{$[\si{\newton\per\meter\squared}] = [\si{\pascal}]$}\]
und für die \textbf{Schubspannung} $\tau$ senkrecht dazu:
\[\tau = \diff {F_\parallel} a \tag*{$[\si{\newton\per\meter\squared}] = [\si{\pascal}]$}\]
Für ein lineares Verhältnis zwischen $\sigma$ und der relativen Auslenkung $\varepsilon_l$ erhalten wir
$$\varepsilon_l = \frac{\Delta l}{l} = \frac{\sigma}{E}$$
wobei $E$ mit $[\si{\pascal}]$ das Elastizitätsmodul ist (i.e. wie resistent es gegenüber Verformungen ist). Für die Phasengeschwindigkeit von longitudinalen Wellen in Festkörpern erhalten wir
$$v = \sqrt{\frac{E}{\rho}}$$

\subsubsection{Ebene Wellen \& Polarisation}

\subsubsection{Kugelwellen}

\subsubsection{Energietransport}

\subsubsection{Intensität}

\subsection{Prinzip der Superposition}
\subsection{Reflexion und Transmission}
\subsection{Stehene Wellen}
\subsection{Akustik, Musikinstrumente*}
\subsection{Beugung, Brechung \& Dispersion}

\subsubsection{Huygen'sche Prinzip}
Betrachte jeden Punkt einer Wellenfront als Quelle einer Kugelwelle. Somit können wir die Beugung an einem Gitter mit $N$ Spalten und Splatabstand $\delta$ herleiten:
$$\xi(\alpha, r, t) = \frac{a}{r}\cdot\frac{\sin\bk{N \frac{\Delta \varphi}{2}}}{\sin\bk{\frac{\Delta \varphi}{2}}} \cdot e^{i\bk{kr-\omega t}}$$
wobei $\alpha$ der Betrachtungswinkel, $r$ die Distanz zum Schirm ist ($d \ll r$ für Kleinwinkelnäherung) und $$\Delta\varphi = k\delta \sin \alpha$$ die Phasendifferenz zwischen zwei benachbarten Punktquellen von Amplitude $a$ ist. Für die Intensität erhalten wir:
$$\abk{I} \sim \frac{a^2}{r^2} \cdot \frac{\sin^2\bk{N \frac{\Delta \varphi}{2}}}{\sin^2\bk{\frac{\Delta \varphi}{2}}}$$
Es gilt also:
\begin{itemize}
    \item Bei $\alpha = 0$ ist ein Maximum, dessen Breite mit grösserem $N$ abnimmt.
    \item Für $\delta > \lambda$ bestimmt $\frac{\delta}{\lambda}$ den Abstand der Maxima: $\sin \alpha_n = n \frac{\lambda}{\delta}$ mit $n \in \N$
\end{itemize}

\subsubsection{Beugung am Einzelspalt}
Bilden wir den Limes $N \to \infty$ und $\delta \to 0$, erhalten wir die Beugung am Spalt der Breite $d = \delta \cdot N$ und Intensität $A = N \cdot a, (a \to 0)$:
$$\abk{I} \sim A^2\frac{\sin^2 \frac{\Delta \varphi}{2}}{\bk{\frac{\Delta \varphi}{2}}^2}$$
Die Intensitätsminima $\abk{I} = 0$ liegen somit bei:
$$ d \sin \alpha = n \lambda$$
Also führen ein grösseres $d$ oder kürzeres $\lambda$ zu einem engeren Beugungsmuster. Es folgt:
\begin{itemize}
    \item $d < \lambda$: Breites Maximum in der Mitte, für $d \ll \lambda$ erhalten wir eine Punktquelle.
    \item $d \approx \lambda$: Beugung mit mehreren Maxima.
    \item $d \gg l$: Schattenwurf, also keine Beugung resp. Inteferenz.
\end{itemize}

\subsubsection{Reflexion \& Brechung}
Geometrisch druch das Huygen'sche Prinzip motiviert erhalten wir für die \textbf{Reflexion}:
$$\alpha_{\text{Eintritt}} = \alpha_{\text{Austritt}}$$
wobei die Winkel zur Normalen der Reflexionsfläche gemessen sind. Ähnlich folgt das \textbf{Brechungsgesetz}, resp. Snellius Gesetz mit unterschiedlichen Phasengeschwindigkeiten $v_1, v_2$ von verschiedenen Medien:$$\frac{\sin \alpha}{\sin \beta} = \frac{v_1}{v_2} = \frac{\lambda_1}{\lambda_2}$$
Das Fermat'sche Prinzip, das Prinzip der kürzesten Laufzeit, liefert dasselbe Resultat. Eine \textbf{Totalreflexion} tritt auf bei:
$$\sin \alpha_2 = \frac{v_2}{v_1}\sin \alpha_1 > 1$$
da Werte über 1 nicht im reellen Definitionsbereich von $\arcsin$ sind.
\subsubsection{Dispersion*}
Die Annahme ohne Dispersion war $v_p = \frac{\omega}{k} = \text{const}$. Mit Dispersion erhalten wir die Dispersionsrelation $v(k) = \frac{d\omega(k)}{dk}$ für verschiedene Frequenzen. Also gilt:
$$\xi(x,t) = \frac{1}{\sqrt{2}} \int_{-\infty}^\infty A(k) e^{i (kx - \omega(k)t)}dk$$
Ein Wellenpaket (Gemisch aus vielen Frequenzen nach Fourier-Transformation) hat somit eine Gruppengeschwindigkeit $v_g$ und eine Phasengeschwindigkeit $v_p$
    \section{Elektrostatik}
Betrachtung statischer elektrischer Ladungen
\subsection{elektrische Ladung}
\begin{itemize}
    \item Ladung $q$ positiv oder negativ
    \item Ladungsträger: Elektronen (-) und Protonen (+)
    \item Elementarladung $q_e$
    \item Einheit: Coulomb $1\si{\coulomb} = 6.25\cdot10^{18} q_e$
    \item Erhaltung der Ladung
\end{itemize}

Ladungsdichte: $\lambda = \diff qx$, Flächenladungsdichte $\tau = \diff qA$, Raumladungsdichte $\rho = \diff qV$

\subsection{Coulombsche Gesetz}
Die elektrostatische Kraft von 1 auf Körper 2 ist
\[F_{21} = k \frac{q_1q_2}{ (r_{21})^2}\evec{r}_{21}\tag*{[\si{\newton}]}\]
mit Coulomb-Konstante $k = \frac{1}{4\pi \varepsilon_0}$ (im SI-System)

Elektrostatische Kraft ist $2\cdot 10^{39} = \frac{F_e}{F_g} = \frac{k}{G}$ stärker als die Gravitationskraft.

\subsection{Energie einer Ladungsverteilung}
Die Kraft $F_{21}$ über den Radius $\infty \to r_{21}$ integriert gibt
\[E_{pot} = W = -\int_\infty^{r_{21}}\vec{F}_{21}(r)\ d\vec{s} = \frac{1}{4\pi \varepsilon_0}\frac{q_1q_2}{r_{21}}\unit{\joule}\]
Für $n$ Ladungen gilt:
\[W_{tot} = \frac{1}{4\pi \varepsilon_0}\frac{1}{2}\sum_{j, i\neq j} \frac{q_j q_i}{r_{ji}}\unit{\joule}\]
Da $\vec{F}_{21} \cdot d\vec{s}_\perp = 0$ gilt, ist das Arbeitsintegral wegunabhängig. Dies gilt immer für konservative Kraftfelder.

\subsection{Das elektrische Feld}
Das elektrische Feld bezeichnet die Kraft auf eine Testladung $q_0$ (keine Influenz: $q_0$ sehr klein). Wir erhalten ein Vektorfeld:
\[\vec{E}(\vec{r}_0) = \frac{1}{4\pi \varepsilon_0} \sum_{i} \frac{q_i}{\abs{\vec{r}_0 - \vec{r}_i}^3}\bk{\vec{r}_0 - \vec{r}_i} \unit{\newton \per \coulomb}\]
Für kontinuierliche Verteilungen mit der Ladungsdichte $\rho$ gilt:
\[\vec{E}(\vec{r}) = \frac{1}{4\pi \varepsilon_0} \int_{\R^3} \frac{\rho}{\abs{\vec{r} - \vec{r'}}^3}\bk{\vec{r}_0 - \vec{r'}}dV \unit{\newton \per \coulomb}\]
wobei die Ladungsdichte $\rho = \rho(\vec{r'})$ vom der Position abhängen kann. Für einen unendlichen Stab der Ladungsdichte $\lambda$ ist das elektrische Feld gegeben durch
\[E = \frac{2k\lambda}{r}\]
\subsubsection{Feldlinien}
Feldlinien laufen (in der Elektrostatik) von $+$-Ladungen zu $-$-Ladungen, also liegen die Kraftvektoren von $\vec{E}$ tangent zu den Feldlinien. Die Stärke des Feldes wird qualitativ durch die Dichte der Feldlinien dargestellt.
\subsection{Gauss'sches Gesetz}
Der elektrische Fluss $\Phi_E$ ist die Feldliniendichte über die Oberfläche eines Volumes $\partial V$:
\[\Phi_E = \vec{E} \cdot \vec{A} = \oint_{\partial V}\vec{E}\cdot d\vec{A}\unit{\volt\meter}\]
Für eine Punktladung $q$ erhalten wir mit dem Integral über die Kugeloberfläche:
\[\Phi_E = \frac{q}{\varepsilon_0}\]
Da $\vec{E} \cdot d\vec{A}_\perp = 0$ keinen Beitrag zum Fluss liefert, gilt unabhängig von der Form von $V$ für Ladungen $q_i \in V$:
\[\Phi_E = \frac{1}{\varepsilon_0}\int_{V}\rho \ dV \unit{\volt\meter}\]

\subsection{Felder von einfachen Ladungsverteilungen}
\subsubsection{Kugeloberfläche}
Das elektrische Feld einer mit $Q$ geladenen Kugel mit Radius $R$ und Flächenladungsdichte $\sigma = \frac{Q}{4\pi R^2}$ ist
\begin{itemize}
    \item $r>R$: $$\oint_{\partial V = 4\pi r^2}\Vec{E}\cdot \Vec{A} = E(r) 4\pi r^2 = \frac{Q}{\varepsilon_0}$$
    also gilt
    $$\Vec{E}(\Vec{r}) = \frac{1}{4\pi \varepsilon_0} = \frac{Q}{r^2}\evec{r}$$
    \item $r<R$: Da sich die Ladungen auf gegenüberliegenden Kugelsegmenten durch jeden Punkt $\in V$ aufheben, gilt $$\Vec{E}(\Vec{r}) = 0$$
\end{itemize}

\subsubsection{langer Stab}
Das elektrische Feld eines mit $Q$ geladenen Stabs mit Länge $l \to \infty$ und Längenladungsdichte $\lambda = Q \cdot l$ ist
$$E(r) = \frac{\lambda}{2\pi \varepsilon_0 r} = \frac{2k\lambda}{r} $$

\subsubsection{ausgedehnte Ebene}
Das elektrische Feld einer mit $Q$ geladenen Fläche mit Radius $R$ und Flächenladungsdichte $\sigma = \frac{Q}{A}$ ist homogen:
$$E(r) = \frac{\sigma}{2 \varepsilon_0}$$
Für einen Plattenkondensator mit Ladungen $Q$ und $-Q$ erhalten wir zwischen den Platten
$$E(r) = \frac{\sigma}{\varepsilon_0}$$
und ausserhalb der Platten gilt $E(r) = 0$.

\subsection{Energie im elektrischen Feld}
Die \textbf{Energiedichte} eines elektrischen Feldes ist gegeben durch 
\[u = \frac{\varepsilon_0}{2}E^2\unit{\joule\per \meter \cubed}\]
Die gesamte elektrische potenzielle Energie ist also
\[U = \int_V \frac{\varepsilon_0}{2}E^2 dV \unit{\joule}\]

\subsection{elektrische Potential}
Es gilt $\Vec{F}_E= q_0\Vec{E} = -\Vec{F}_{ext}$ für quasi-stationäres Kräftegleichgewicht:
\begin{multline*}
    W_{BA} = W_{ext} = \int_A^B\Vec{F}_{ext} \cdot d\Vec{s}\\ = -q_0 \int_A^B \Vec{E}\cdot d\Vec{s} = U_B - U_A \unit{J}
\end{multline*}
Es ist ein konservatives Vektorfeld, also gilt für Kreisintegrale:
$$\oint \vec{E}d\vec{s} = 0 \Leftrightarrow \vec{E} = -\vec{\nabla}\phi$$
daraus folgt
\[\int_A^B\vec{E}d\vec{s} = \phi_B - \phi_A =: \phi_{BA} \unit{\joule \per \coulomb = \volt}\]
Notation:
\begin{itemize}
    \item $\Phi$ Potential
    \item $V$ Spannungsquelle (Voltage) mit $V = \Phi_B - \Phi_A$
    \item $U$ potentielle Energie
\end{itemize}

Für allgemeine Ladungsverteilungen gilt:
\[\phi(\vec{r}) = \frac{1}{4\pi \varepsilon_0}\int_{\R^3}\frac{\rho(\vec{r'})}{\abs{\vec{r}-\vec{r'}}} dV \unit{\volt}\]


\subsection{Potentiale von einfachen Ladungsverteilungen}

\subsubsection{Plattenkondensator}
\[\phi_{BA} = E \Delta z \unit{V}\]
\[W_{BA} = E q_0 \Delta z \unit{J}\]
\subsubsection{Punktladung}
\[\phi(r) = \frac{1}{4\pi \varepsilon_0}\frac{q}{r}\unit{V}\]

\subsubsection{Ringladung}
\[\phi(x_0) = \frac{1}{4\pi \varepsilon_0}\frac{q}{\sqrt{x_0^2+R^2}}\unit{V}\]

\subsection{Satz von Gauss}
Für ein Vektorfeld $\Phi$ gilt
\[\int_{\partial V} \vec{F} \cdot d \vec{a} = \int_{V}\divergence \vec{F} \cdot dV\]
Wir können also ein Flächenintegral in ein Volumenintegral überführen.
Auf das Gauss'sche Gesetz angewendet erhalten wir
\[\divergence \vec{E} = \frac{\rho}{\varepsilon_0}\]

\subsection{Satz von Stokes}

    \section{Elektrische Leiter}
\begin{itemize}
    \item Kein $\Vec{E}$ in Leitern
    \item $\rho = 0$ im Leiter 
    \item Ladungen auf Oberfläche
    \item Potential $\phi = $ konst im Leiter 
\end{itemize}

\subsection{Faradaysche Käfig}

\subsection{Tricks}

\subsection{Kondensatoren}
\[Q = C V \unit{\coulomb}\]

\subsubsection{Plattenkondensator}
\[C = \frac{A }{\varepsilon_0d} \unit{\coulomb \per \volt}\]
\[W = \frac{1}{2} C V^2= \frac{1}{2}Q V \unit{\joule}\]

Klemmenspannung $\phi_{BA} = V-V_0=V$
Elektromotorische Kraft $\varepsilon \approx V$
Ladunsträgerdichte (Anzahl beweglicher Ladungen pro Volumen) $[cm^{-3}]$

\subsubsection{Stromstärke}
\[I = \quer{v} A n q_e \unit{\ampere}\]
Stromrichtung: Konventionelle (Richtung der $+$), physikalische (Richtung der $-$). Reelle/reale Leiter: Reibung/Dissipation der Energie, externes Feld benötigt. Ideale Leiter: ''Supraleiter'' $\Vec{v} = \text{konst}$ mit $\Vec{E}_{ext} = 0$ 

\subsubsection{Stromdichte Vektor}
\[\Vec{J} = \frac{I}{A} = n q \Vec{v}\]
Stromdichte durch gerichtete Fläche
\[\Vec{I}_A = \int_A\Vec{J}\cdot d \Vec{A}\]

Kontinuitätsgleichung: 
\[\Vec{\nabla}\cdot\Vec{J} = - \diffp{\rho}{t} dV\]
Hergeleitet aus
\[I_{\partial V} = \int_V -\diffp{\rho}{t} dV = \]

\subsubsection{Leitfähigkeit}
\[\Vec{J} = \omega \Vec{E}\]
$\omega$ kann Skalar oder Matrix sein (Richtungsabhängigkeit von Leitern). Der \textbf{Spezifische Widerstand} ist $\rho = \frac{1}{\omega}$ in $[\Omega]$

\subsection{Kirchoffschen Regeln}
\begin{enumerate}
    \item Für jedes Element gilt $V_i = R_iI_i$
    \item Für jeden Knoten gilt $\sum_i I_i = 0$
    \item Für jede Masche gilt $\sum_i V_i = 0$
\end{enumerate}
    
    \section{Mathematische Werkzeuge}
\subsection{Unsicherheiten}
Bei einer absoluten Unsicherheit $\delta u$  der Messgrösse $u$ ist die relative Unsicherheit
$$\frac{\delta u}{u}$$

\subsubsection{Fehlerfortpflanzung}
Für eine Funktion $f(x_1, x_2, ...)$ gilt
$$\delta f = \abs{\diffp{f}{x_1}}\delta x_1 + \abs{\diffp{f}{x_2}} \delta x_2 + ...$$

\subsection{Vektoranalysis}
\textbf{Nabla-Operator} (Notation):
$$\Vec{\nabla} = \begin{pmatrix}\diffp{}{x}& \diffp{}{y} & \diffp{}{z}\end{pmatrix}^T$$
\textbf{Laplace-Operator}: für ein Skalarfeld $\phi$ gilt
\[\Delta \phi = \nabla\cdot(\nabla \phi)\]
also gilt
\[\Delta \phi = \diffp[2]{\phi}{x} + \diffp[2]{\phi}{x} + \diffp[2]{\phi}{z}\]

\subsubsection{Skalarfeld}
Ein \textbf{Skalarfeld} (z.B Gravitationspotential) teilt jedem Punkt einen Skalar zu:
$$\Phi: \R^3 \to \R$$
Der \textbf{Gradient} von einem Skalarfeld ist der Richtungsvektor des ''stärksten Anstiegs'':
$$\text{grad} \Phi = \Vec{\nabla}(\Phi) = \begin{pmatrix}\diffp{\Phi}{x}& \diffp{\Phi}{y} & \diffp{\Phi}{z}\end{pmatrix}^T \in \R^3$$
Es gilt im allgemeinen
$$\Vec{\nabla} \cdot (\Vec{\nabla} \times \Vec{A}) =  = 0$$

\subsubsection{Vektorfeld}
Ein \textbf{Vektorfeld} (z.B. Gravitationsfeld) teilt jedem Punkt einen Vektor zu:
$$\Vec{F}: \R^3 \to \R^3$$
Die \textbf{Divergenz} eines Vektorfeldes $\divergence\Vec{F}: \R^3 \to \R$ ist
$$\divergence \Vec{F} = \Vec{\nabla} \cdot \Vec{F} = \diffp{\Vec{F}}{x} + \diffp{\Vec{F}}{y} + \diffp{\Vec{F}}{z} \in \R$$
Sie bezeichnet die Quellstärke von $\Vec{F}$ ($\divergence \Vec{F}(x) < 0$: Senke, $=0$ quellenfrei, $>0$ Quelle)

Die \textbf{Rotation} eines Vektorfeldes $\rot\Vec{F}: \R^3 \to \R^3$ ist
$$\rot \Vec{F} = \Vec{\nabla} \times \Vec{F} =  \begin{pmatrix}\diffp{F_z}{y}-\diffp{F_y}{z}\\ \diffp{F_x}{z}-\diffp{F_z}{x} \\ \diffp{F_y}{x}-\diffp{F_x}{y}\end{pmatrix} \in \R^3$$
Sie bezeichnet nach der Rechten-Hand-Regel die Rotationsachse von Wirbeln im Feld.

Wir nennen ein Feld \textbf{konservativ}, falls es wirbelfrei ist: $\rot\Vec{F} = 0$. Das \textbf{Gradientenfeld} $\grad \Phi: \R^3 \to \R^3$ eines \textit{Skalarfeldes} ist ein konservatives Vektorfeld.

\subsubsection{Linienintegral im Skalarfeld}
Für ein Skalarfeld $\Phi: \R^3 \to \R$
\subsubsection{Linienintegral im Vektorfeld}
Für ein Vektorfeld $\Vec{F}:\R^3 \to \R^3, \Vec{r} \mapsto \Vec{F}(\Vec{r})$ und ein Weg $\gamma: [a,b] \to \R^3$ erhält man durch Substitution:
$$\int_\gamma \Vec{F}(\Vec{r}) d\Vec{r} = \int_a^b \Vec{F}(\gamma(x)) \diff{\gamma}{x} dx$$
In $\R^2$ gilt für $\gamma(x) = \binom{f(x)}{g(x)}$:
\begin{align*}
    \int_\gamma \Vec{F}(\Vec{r}) d\Vec{r} &= \int_a^b \Vec{F}\binom{f(x)}{g(x)}\cdot\diff{\Vec{r}}{x} dx\\
    &= \int_a^b F_x(f(x))f'(x) + F_y(g(x)) g'(x) dx 
\end{align*}
Für konservative Vektorfelder ($\Vec{\nabla} \cdot F = 0, F = 0$) gilt über eine Fläche $A$
$$\oint_{\partial A} \Vec{F} \cdot d\Vec{s} \stackrel{\text{Stokes}}{=} \int_{A} \rot \Vec{F} d\Vec{a} = 0$$
\subsubsection{Kugelflächenintegral}
Winkelelement $d\Omega = \sin \vartheta\ d\vartheta\ d\varphi$

\end{document}
