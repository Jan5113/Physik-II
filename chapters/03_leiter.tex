\section{Elektrische Leiter}
\begin{itemize}
    \item Kein $\Vec{E}$ in Leitern
    \item $\rho = 0$ im Leiter 
    \item Ladungen auf Oberfläche
    \item Potential $\phi = $ konst im Leiter 
\end{itemize}

\subsection{Faradaysche Käfig}

\subsection{Tricks}

\subsection{Kondensatoren}
\[Q = C V \unit{\coulomb}\]

\subsubsection{Plattenkondensator}
\[C = \frac{A }{\varepsilon_0d} \unit{\coulomb \per \volt}\]
\[W = \frac{1}{2} C V^2= \frac{1}{2}Q V \unit{\joule}\]

Klemmenspannung $\phi_{BA} = V-V_0=V$
Elektromotorische Kraft $\varepsilon \approx V$
Ladunsträgerdichte (Anzahl beweglicher Ladungen pro Volumen) $[cm^{-3}]$

\subsubsection{Stromstärke}
\[I = \quer{v} A n q_e \unit{\ampere}\]
Stromrichtung: Konventionelle (Richtung der $+$), physikalische (Richtung der $-$). Reelle/reale Leiter: Reibung/Dissipation der Energie, externes Feld benötigt. Ideale Leiter: ''Supraleiter'' $\Vec{v} = \text{konst}$ mit $\Vec{E}_{ext} = 0$ 

\subsubsection{Stromdichte Vektor}
\[\Vec{J} = \frac{I}{A} = n q \Vec{v}\]
Stromdichte durch gerichtete Fläche
\[\Vec{I}_A = \int_A\Vec{J}\cdot d \Vec{A}\]

Kontinuitätsgleichung: 
\[\Vec{\nabla}\cdot\Vec{J} = - \diffp{\rho}{t} dV\]
Hergeleitet aus
\[I_{\partial V} = \int_V -\diffp{\rho}{t} dV = \]

\subsubsection{Leitfähigkeit}
\[\Vec{J} = \omega \Vec{E}\]
$\omega$ kann Skalar oder Matrix sein (Richtungsabhängigkeit von Leitern). Der \textbf{Spezifische Widerstand} ist $\rho = \frac{1}{\omega}$ in $[\Omega]$

\subsection{Kirchoffschen Regeln}
\begin{enumerate}
    \item Für jedes Element gilt $V_i = R_iI_i$
    \item Für jeden Knoten gilt $\sum_i I_i = 0$
    \item Für jede Masche gilt $\sum_i V_i = 0$
\end{enumerate}